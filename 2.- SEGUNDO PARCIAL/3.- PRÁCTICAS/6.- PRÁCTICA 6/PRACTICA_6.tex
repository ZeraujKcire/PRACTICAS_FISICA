\documentclass[12pt,a4paper]{article}

% === PAQUETES === (((
\usepackage{amsmath}
\usepackage{colortbl}
\usepackage{amsfonts}
\usepackage{multicol}
\usepackage{multirow}
\usepackage{xcolor}
\usepackage{amssymb}
\usepackage{listings} 
% \usepackage{expl3}
\usepackage{fontspec}
\usepackage{fullpage}
\usepackage{graphicx}
\usepackage{titlesec} 
% \usepackage{setspace}
\usepackage{dsfont}
% \usepackage{bookmark}
% )))

% === TIPOGRAFÍA === (((
\setmainfont[
  BoldFont       = bodonibi,
	ItalicFont     = Century modern italic2.ttf,
	BoldItalicFont = bodonibi,
	SmallCapsFont  = lmromancaps10-regular.otf
]{Century_modern.ttf}
% )))

% === COMANDOS === (((
\newcommand{\dis}{\displaystyle}
\newcommand{\qed}{\hspace{0.5cm}\rule{0.16cm}{0.4cm}}
\newcommand{\micita}[1]{\([\)\cite{#1}\(]\)}
\newcommand{\operator}[1]{\mathop{\vphantom{\sum}\mathchoice{ \vcenter{\hbox{\huge $#1$}} }
{\vcenter{ \hbox{\Large $#1$}} }{#1}{#1}}\displaylimits}
\newcommand{\suma}{\operator{ \includegraphics[scale=0.09]{IMAGENES/Sigma.png}} }
\DeclareSymbolFont{italics}{\encodingdefault}{\rmdefault}{m}{it}
\DeclareSymbolFontAlphabet{\mathit}{italics}
\ExplSyntaxOn
\int_step_inline:nnnn { `A } { 1 } { `Z }
 {  \exp_args:Nf \DeclareMathSymbol{\char_generate:nn{#1}{11}}{\mathalpha}{italics}{#1} }
\int_step_inline:nnnn { `a } { 1 } { `z } {  \exp_args:Nf \DeclareMathSymbol{\char_generate:nn{#1}{11}}{\mathalpha}{italics}{#1}}
\ExplSyntaxOff
% )))

% === SECCIONES === (((
\titleformat*{\section}{\large\normalfont\bfseries}
\titleformat*{\subsection}{\large\itshape}
\titleformat*{\subsubsection}{\bfseries\centering}
% \setcounter{secnumdepth}{0}
\renewcommand*{\contentsname}{\large\textbf{CONTENIDOS.}}
\usepackage[nottoc,numbib]{tocbibind}
\renewcommand{\refname}{REFERENCIAS.}
\renewcommand{\tablename}{Tabla}
\renewcommand{\figurename}{Figura}
% )))

% === PORTADA === (((
% \pagestyle{empty}
\newcommand{\portada}{
\addfontfeature{LetterSpace=-5}
  \begin{titlepage}
  \centering
  \begin{figure}
    \centering
    \includegraphics[scale=0.5]{IMAGENES/logo_uaa.png}  
  \end{figure}
  {\bfseries\Large\MakeUppercase{\textit{Universidad Autónoma de Aguascalientes.}} \par}
  \vspace{1cm}
  {\Large Centro de Ciencias Básicas. \vspace{0.5cm}\\[2mm]
  Departamento de Matemáticas y Física.\vspace{0.5cm}\\[2mm]
  Licenciatura en Matemáticas Aplicadas.\vspace{0.5cm}\\[2mm]
  Práctica 7.\par}
  \vspace{1.5cm}
  {\bfseries\Huge Difracción. \par} % title
  \vspace{1.5cm}
  {\itshape\Large Óptica. \\Prof. Mariana Alfaro Gómez.\par}
  % {\itshape\Large Variable Compleja I. \\Prof. Fausto Arturo Contreras Rosales.\par}
  % {\itshape\Large Métodos Numéricos II. \\Prof. Manuel Ramírez Aranda.\par}
  % {\itshape\Large Diseño de Experimentos. \\Prof. Angélica Hernández Quintero.\par}
  % {\itshape\Large Filosofía de la Investigación Científica. \\Prof. Jesús Mariano Rodríguez Muñoz.\par}
  \vfill
  % {\Large \textit{Por Erick I. Rodríguez Juárez.}\par}
		\begin{flushleft}
		\Large
		Alumnos:\\
		\textit{Carlos Francisco Guzmán Barba.}\\
		\textit{Erick Ignacio Rodríguez Juárez.}\\
		\textit{Manuel Alejandro Siller Landin.}
		\end{flushleft}
	% {}  % {\Large \textit{Por Erick I. Rodríguez Juárez.}\par}
  \vfill
		\begin{flushright}
		{\Large Realización: 16\(/\)05\(/\)22. \par} % date
		{\Large Entrega: 23\(/\)05\(/\)22. \par} % date
		\end{flushright}
  \end{titlepage} 
	% \thispagestyle{empty}
	% \doublespacing
	% \tableofcontents
	% \singlespacing
	% \newpage
} 
% )))

% === LST-LISTINGS PARA VER EL CÓDIGO === (((
\usepackage{xcolor}
\definecolor{backcolour}{rgb}{0.95,0.95,0.92}

\lstdefinestyle{mystyle}{
  backgroundcolor  =  \color{backcolour},
  commentstyle     =  \color{gray},
  numberstyle      =  \tiny\color{gray},
  stringstyle      =  \color{purple},
  basicstyle       =  \ttfamily\footnotesize,
  numbers          =  left,
  numbersep        =  5pt,
  keywordstyle     =  \color{blue},
  identifierstyle  =  \color{orange},
}

\lstset{style=mystyle}
% )))

\begin{document}

\portada

\section{RESUMEN.} % (((
% )))

\section{INTRODUCCIÓN.} % (((

\subsection{--- Difracción en una Rendija Rectangular ---} % (((
\label{sub:difraccion_una}
\begin{figure}[hbtp!]
\addfontfeature{LetterSpace = -5}
\begin{minipage}{0.55\linewidth}
	\textbf{Definición.} \textit{En el movimiento de propagación de un sistema de ondas se le llama \textbf{difracción} al proceso experimental en el que chocan contra una pared con una abertura proporcional (rectangular , o circular) a la longitud de la onda.} \\[2mm]
 	En tal experimento, se tiene la siguiente situación.
	Las ondas que impactan la pantalla se relfejan detrás de ella.
	Aquellas que pasen por la abertura rectangular, se refractarán en todas las direcciones posibles, como lo indica la Figura \ref{fig:rectang} (tomada de \micita{hecht}).
\end{minipage}\hspace{5mm}
\begin{minipage}{0.45\linewidth}
	\includegraphics[width= 0.9 \linewidth]{1_INTRO/direcciones.png}
	\caption{Propagación de la onda.}
	\label{fig:rectang}
\end{minipage}
\end{figure}
\begin{figure}[hbt!]
	\addfontfeature{LetterSpace = -5}
	\begin{minipage}{0.4\linewidth}
	\centering
	\includegraphics[width= \linewidth]{1_INTRO/seno.png}
	\caption{Difracción de la onda en la dirección del ángulo \(\theta\).}
	\label{fig:seno}
	\end{minipage}\hspace{5mm}
	\begin{minipage}{0.6\linewidth}
		Entonces las ondas que atraviesen la rendija alcanzarán un máximo en la onda, en aquellos ángulos \(\theta\), tal que las continuaciones de los rayos coincidan con los máximos antes de golpear la rendija. Como lo indica la Figura \ref{fig:seno} (obtenida de \micita{alonso_finn_1}).
		Es decir, si \(d\) es el tamaño de la rendija, \(\lambda\) la longitud de onda, y \(\theta\) el ángulo de desviación, tendremos:
		\begin{equation}
			d \sin \theta = m \lambda , \hspace{1cm} m \in \mathds{Z}.
			\label{eq:posicion_angular}
		\end{equation}
	\end{minipage}
\end{figure}
\begin{figure}[hbtp!]
	\addfontfeature{LetterSpace = -5}
	\begin{minipage}{0.5\linewidth}
		Además, notamos que si 
		% el ángulo \(\theta\) es suficientemente pequeño,
		\(y\) es la distancia del máximo central al \(m-\)ésimo máximo de la ec. (\ref{eq:posicion_angular}), y \(D\) es la separación de una pantalla con la rendija, entonces se tendrá \(\sin \theta = \tan \theta\), y
		\begin{equation}
			\tan \theta = \dfrac{y}{D}.
			\label{eq:distan_rendija}
		\end{equation}
		Así, como lo indica la Figura \ref{fig:constructiva}. Combinando (\ref{eq:posicion_angular}) y (\ref{eq:distan_rendija}), obtenemos que
		\begin{equation}
			y = \dfrac{m \lambda D}{d} , \hspace{1cm} m \in \mathds{Z} .
			\label{eq:altura}
		\end{equation}
	\end{minipage}\hspace{5mm}
	\begin{minipage}{0.5\linewidth}
		\centering
		\includegraphics[width= 0.8 \linewidth]{1_INTRO/distancia}
		\caption{Tamaño de la difracción constructiva.}
		\label{fig:constructiva}
	\end{minipage}
\end{figure}
% )))

\subsection{--- Difracción de Doble Rendija ---} % (((
\label{sub:difraccion_dos}
Ahora, consideremos dos rendijas, ambas de tamaño \(a\), separadas a una distancia \(d\), como lo indica la Figura \ref{fig:exp_young}, al cual se le conoce como \textbf{Experimento de Young}.
\begin{figure}[hbt!]
	\centering
	\includegraphics[width= 0.8 \linewidth]{1_INTRO/dos_rendijas.png}
	\caption{Experimento de Young.}
	\label{fig:exp_young}
\end{figure}\\
Notamos que la diferencia de fase entre ambas ondas es de
\[
	\alpha = \dfrac{2 \pi}{\lambda} CE = \dfrac{2 \pi d \sin \theta}{\lambda}.
\]
\begin{figure}[hbtp!]
	\addfontfeature{LetterSpace = -5}
	\begin{minipage}{0.5\linewidth}
	Notamos que, para la difracción de Faunhofer, se tiene la situación de la Figura \ref{fig:radios}. \\[2mm]
	Entonces, \(r \approx R - y \sin \theta\), y por cada rendija se tiene que 
		\[
			F(y) = \sin (wt- kr) = \sin (wt-k(R - y \sin \theta))
		\]
		Y también, para cada rendija tendremos la siguiente expresión para el campo eléctrico,
		\[
			E = C \dis\int _{y_{min}} ^{y_{max}} F(y) dy.
		\]
		Indicando que, en nuestro caso con dos rendijas, y con las longitudes de la Figura \ref{fig:exp_young}, obtendremos que
	\end{minipage}\hspace{5mm}
	\begin{minipage}{0.5\linewidth}
	\centering
	\includegraphics[width= 0.8 \linewidth]{1_INTRO/fran}
	\caption{Radios hacia la rendija.}
	\label{fig:radios}
	\end{minipage}
\end{figure}
\begin{equation}
	\begin{array}{rcl}
		E & = & C \dis\int _{-a/2} ^{a/2} F(y) dy + C \dis\int _{d-a/2} ^{d+a/2}  \\[5mm]
		& = & bc \bigg(\dfrac{\sin \beta}{\beta}\bigg) \big(\sin (wt-kR) + \sin (wt-kR + \alpha)\big)
	\end{array}
	\label{eq:campo_electrico}
\end{equation}
donde \(\beta = \dfrac{kb}{2} \sin \theta\), y \(\alpha = \dfrac{ka}{2} \sin \theta\). La gráfica de ésta función se deja indicada en la Figura \ref{fig:grafica}. 
\begin{figure}[hbt!]
	\centering
	\includegraphics[width= 0.7 \linewidth]{1_INTRO/grafica}
	\caption{Gráfica de la función (4).}
	\label{fig:grafica}
\end{figure}\\
Además, se tiene \(I= \langle E \rangle ^2/2 = 4I_0 \bigg(\dfrac{\sin ^2 \beta}{\beta ^2}\bigg) \cos ^2\alpha \), se le llama irradiancia ``normalizada'' a la función \(I/(4I_0)\).
% )))

% )))

\section{METODOLOGÍA.} % (((
% )))

\section{RESULTADOS.} % (((
% )))

  La incertidumbre del Vernier es de $ \dfrac{0.01 cm}{2}=0.005 cm $
	 
	 La incertidumbre del banco óptico es de $ \dfrac{0.1 cm}{2}=0.05 cm $
	 
	 
	 Durante el experimento, se tenían los siguientes datos ``fijos$"$:
	 
	 \begin{itemize}
	 	\item La distancia $ D $ entre la doble rendija y la pantalla era $ D=(90\pm 0.05) cm$
	 	\item La longitud de la onda del láser empleado era de $ \lambda=650 nm=6.5\times 10^{-5}cm $
	 \end{itemize}
 
 	Luego, para dos máximos de interferencia elegidos $ (m_1, m_2) $ en la pantalla, y considerando un ancho de rendija $ a $ separadas una distancia $ d $, la distancia $ y $ entre tales máximos elegidos se muestran en la siguiente tabla \ref{tab:distancias}
 	
 	\begin{table}[!htb]
 		\centering
 		\caption{Distancias entre máximos observados}
 		\begin{tabular}{|c|c|c|}
 			\hline
 			\backslashbox{$ a, d $ (mm)}{$ (m_1,m_2) $}& $ (0,2) $ & $ (0,4) $ \\
 			\hline
 			$ a=0.04 $ & \multirow{2}{*}{$ (0.395\pm 0.005) cm $} & \multirow{2}{*}{$ (0.845\pm 0.005) cm $} \\ 		
 			$ d=0.25 $ &  & \\ 
 			\hline
 			$ a=0.08 $& \multirow{2}{*}{$ (0.380\pm 0.005) cm $} & \multirow{2}{*}{$ (0.870\pm 0.005) cm $} \\
 			$ d=0.25 $&  &  \\
 			\hline
 			$ a=0.04 $          & \multirow{2}{*}{$ (0.215\pm 0.005) cm $} & \multirow{2}{*}{$ (0.430\pm 0.005) cm $} \\ 		
 			$ d=0.5\phantom{0} $&  &  \\	
 			\hline
 			$ a=0.08 $           & \multirow{2}{*}{$ (0.180 \pm 0.005) cm $} & \multirow{2}{*}{$ (0.420 \pm 0.005) cm $} \\
 			$ d=0.5\phantom{0} $ &  &  \\
 			\hline
 		\end{tabular}
 		\label{tab:distancias}
 	\end{table}
 	
  	Haciendo uso de la ecuación \ref{}, para los mismos valores anteriores se tienen las siguientes distancias teóricas (tabla \ref{tab:disteo} )
 	
 	\begin{table}[!htb]
 		\centering
 		\caption{Distancias teóricas para los máximos}
 		\begin{tabular}{|c|c|c|}
 			\hline
 			$ d $ (mm) & \multicolumn{2}{c|}{$ y_m $ (cm)}  \\
 			\hline
 			0.25 & $ y_2=0.4680\pm 0.0002 $ & $ y_4=0.9360\pm0.0005 $ \\
 			\hline
 			0.5 & $ y_2=0.2340\pm 0.0001 $ & $ y_4=0.4680\pm0.0002 $  \\
 			\hline
 		\end{tabular}
 		\label{tab:disteo}
 	\end{table}
 	
 	A continuación, en la tabla \ref{tab:errores} se calculan los errores absolutos y relativos de las mediciones obtenidas
 	
 	\begin{table}[!htb]
 		\centering
 		\caption{Errores absolutos y relativos}
 		\begin{tabular}{|c|c|c|c|c|}
 			\hline
 			$ a, d $ (mm) & $ y_m $ observado (cm) & $ y_m $ teórico (cm) & Error absoluto (cm) & Error relativo (\%)  \\
 			\hline
 			$ a=0.04 $          & $y_2=0.395\pm 0.005 $ & $y_2=0.4680\pm0.0002 $ & $ 0.073 $ & $ 15.59 $ \\ \cline{2-5}
 			$ d=0.25 $          & $y_4=0.845\pm 0.005 $ & $y_4=0.9360\pm0.0005 $ & $ 0.091 $ & $ \phantom{0}9.72 $ \\ \hline
 			$ a=0.08 $          & $y_2=0.380\pm 0.005 $ & $y_2=0.4680\pm0.0002 $ & $ 0.088 $ & $ 18.80 $ \\ \cline{2-5}
 			$ d=0.25 $          & $y_4=0.870\pm 0.005 $ & $y_4=0.9360\pm0.0005 $ & $ 0.066 $ & $ \phantom{0}7.05 $ \\ \hline
 			$ a=0.04 $          & $y_2=0.215\pm 0.005 $ & $y_2=0.2340\pm0.0001 $ & $ 0.019 $ & $ \phantom{0}8.11 $ \\ \cline{2-5}
 			$ d=0.5\phantom{0} $& $y_4=0.430\pm 0.005 $ & $y_4=0.4680\pm0.0002 $ & $ 0.038 $ & $ \phantom{0}8.11 $ \\ \hline
 			$ a=0.08 $          & $y_2=0.180\pm 0.005 $ & $y_2=0.2340\pm0.0001 $ & $ 0.054 $ & $ 23.07 $ \\ \cline{2-5}
 			$ d=0.5\phantom{0} $& $y_4=0.420\pm 0.005 $ & $y_4=0.4680\pm0.0002 $ & $ 0.048 $ & $ 10.25 $ \\ \hline
 		\end{tabular} 
 		\label{tab:errores}
 	\end{table}
 	 
	\begin{figure}[htbp!]
		\centering
		\subfloat[Foto de la rendija]{\includegraphics[width=1\textwidth]{04_y_25.jpg}\label{fig:A1}}
		\hfill
		\subfloat[Patrón de irradiancia experimental]{\includegraphics[width=1\linewidth,height=9cm]{04_y_25.pdf}\label{fig:A3}}
		\hfill
		\subfloat[Patrón de irradiancia teórico]{\includegraphics[width=1\linewidth,height=9cm]{Irradiancia 1.png}\label{fig:A2}}
		\hfill
		\caption{Patrones de irradiancia para la rendija indicada}
		\label{fig:P1}
	\end{figure}

	\begin{figure}[htbp!]
		\centering
		\subfloat[Foto de la rendija]{\includegraphics[width=1\textwidth, height=4cm]{08_y_25.jpg}\label{fig:A4}}
		\hfill
		\subfloat[Patrón de irradiancia experimental]{\includegraphics[width=1\linewidth,height=8.5cm]{08_y_25.pdf}\label{fig:A5}}
		\hfill
		\subfloat[Patrón de irradiancia teórico]{\includegraphics[width=1\linewidth,height=9cm]{Irradiancia 2.png}\label{fig:A6}}
		\hfill
		\caption{Patrones de irradiancia para la rendija indicada}
		\label{fig:P2}
	\end{figure}

	\begin{figure}[htbp!]
		\centering
		\subfloat[Foto de la rendija]{\includegraphics[width=1\textwidth]{04_y_5.jpg}\label{fig:A7}}
		\hfill
		\subfloat[Patrón de irradiancia experimental]{\includegraphics[width=1\linewidth,height=9cm]{04_y_5.pdf}\label{fig:A8}}
		\hfill
		\subfloat[Patrón de irradiancia teórico]{\includegraphics[width=1\linewidth,height=9cm]{Irradiancia 3.png}\label{fig:A9}}
		\hfill
		\caption{Patrones de irradiancia para la rendija indicada}
		\label{fig:P3}
	\end{figure}
	
	\begin{figure}[htbp!]
		\centering
		\subfloat[Foto de la rendija]{\includegraphics[width=1\textwidth, height=4cm]{08_y_5.jpg}\label{fig:A10}}
		\hfill
		\subfloat[Patrón de irradiancia experimental]{\includegraphics[width=1\linewidth,height=8.5cm]{08_y_5.pdf}\label{fig:A11}}
		\hfill
		\subfloat[Patrón de irradiancia teórico]{\includegraphics[width=1\linewidth,height=9cm]{Irradiancia 4.png}\label{fig:A12}}
		\hfill
		\caption{Patrones de irradiancia para la rendija indicada}
		\label{fig:P4}
	\end{figure}
		
	
\section{DISCUSIÓN DE RESULTADOS Y CONCLUSIONES.} % (((
% )))

% === REFERENCIAS === (((
\bibliography{Referencias}
\bibliographystyle{unsrt}
% )))

\section{APÉNDICE.} % (((
% )))

\subsection{--- Propagación de la Incertidumbre ---}
	
	La propagación de la incertidumbre para el producto y un cociente están dadas (respectivamente) por
	\begin{align*}
		(x\pm\delta x)(y\pm\delta y)&=x\cdot y\pm\left(|y|\delta x+|x|\delta y \right)\\\\
		\dfrac{x\pm\delta x}{y\pm\delta y}&=\dfrac{x}{y}\pm\left(\dfrac{\delta x}{|y|}+|x|\dfrac{\delta y}{|y|^2}\right)
	\end{align*}
	
	Para el caso particular en que $ y $ no posee incertidumbre, estas operaciones se simplifican a
	\begin{align*}
		(x\pm\delta x)(y)&=x\cdot y\pm|y|\delta x\\\\
		\dfrac{x\pm\delta x}{y}&=\dfrac{x}{y}\pm\dfrac{\delta x}{|y|}
	\end{align*} 

	\subsection{--- Cálculo teórico de las distancias entre máximos ---}	
	
	Haciendo uso de la fórmula \ref{} y los datos experimentales, se obtienen los siguientes cálculos
	\begin{itemize}
		\item Para $ d=0.25mm=0.025cm $

		\begin{align*}
			y_2&=\dfrac{2\cdot (6.5\times10^{-5}cm)\cdot[(90\pm 0.05) cm]}{0.025 cm}=
			\dfrac{(0.0117\pm 6.5\times 10^{-6}) cm^2}{0.025 cm}=
			(0.4680\pm 0.0002)cm\\\\
			y_4&=\dfrac{4\cdot (6.5\times10^{-5}cm)\cdot[(90\pm 0.05) cm]}{0.025 cm}=
			\dfrac{(0.0234\pm1.3\times 10^{-5})cm^2}{0.025 cm}=
			(0.9360\pm0.0005)cm\\
		\end{align*}
	
		\item Para $ d=0.5mm=0.05cm $
		
		\begin{align*}
			y_2&=\dfrac{2\cdot (6.5\times10^{-5}cm)\cdot[(90\pm 0.05) cm]}{0.025 cm}=
			\dfrac{(0.0117\pm 6.5\times 10^{-6}) cm^2}{0.05 cm}=
			(0.2340\pm 0.0001)cm\\\\
			y_4&=\dfrac{4\cdot (6.5\times10^{-5}cm)\cdot[(90\pm 0.05) cm]}{0.025 cm}=
			\dfrac{(0.0234\pm1.3\times 10^{-5})cm^2}{0.05 cm}=
			(0.4680\pm0.0002)cm\\
		\end{align*}	
		 
	\end{itemize}



\end{document}
